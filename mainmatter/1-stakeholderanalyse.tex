\chapter{Stakeholderanalyse}
In 2013 startte Greentom in Zuid-Limburg met één ambitie: een kinderwagen
ontwerpen die zowel duurzaam als gebruiksvriendelijk is. Het was het begin van
een reeks innovatieve ideeën die samenkwamen in een uniek product
\parencite{GreentomOnzeMisse}. Het product legt niet alleen de nadruk op
duurzame materialen, maar ook op de principes repareerbaarheid, veelzijdigheid
en modulariteit \parencite{GreentomDuurzaamheid}. Zo kan de kinderwagen
eenvoudig door de consument in slechts vijf minuten gebouwd worden, een
werkelijk reuzeverschil met de gemiddelde twee uur van concurrenten
\parencite{SmitsmansPresentatieNov2025}.

\section{B Corp-certificaat}
Greentoms toewijding aan duurzaamheid werd verder onderstreept toen het bedrijf
het B Corp-certificaat mocht ontvangen \parencite{GreentomBCorp}. Dit is een
certificaat dat wordt uitgereikt door B-lab, een non-profitorganisatie dat
bedrijven beoordeelt op hun maatschappelijke en ecologische impact
\parencite{BLabAboutUs}. Deze certificering vereist een grondige toetsing:
bedrijven moeten minstens 80 punten behalen in de B Impact Assessment, waarbij
wordt gekeken naar vijf kerngebieden: bestuur, medewerkers, gemeenschappen,
milieu en klanten \parencite{BLabWhat}.
