\chapter{Stakeholderanalyse}
Greentom wordt gekenmerkt door een sterke en helder geformuleerde missie.
De doelstellingen van het bedrijf zijn bewust eenvoudig gehouden, maar vormen
de kern van een bredere visie op duurzaamheid, functionaliteit en
maatschappelijke verantwoordelijkheid \parencite{GreentomOnzeMisse}.
Greentom staat voor ‘Green Tomorrow’ \parencite{GreentomOnzeMisse}. Het draagt
met deze naam de filosofie voort dat het de meest duurzame keuze is om geen
natuurlijke hulpbronnen meer te benutten \parencite{GreentomOnzeMisse}.

Greentom vervult deze doelstellingen door onder andere producten te ontwikkelen
die voor 87\% uit recyclebaar materiaal bestaan
\parencite{SmitsmansPresentatieNov2025}. Het verzorgt daarnaast ook
partnerschappen met lease-bedrijven waardoor consumenten haar producten niet
alleen kunnen kopen maar ook kunnen huren.

\section{Eerste wieltjes}
In 2013 startte Greentom in Zuid-Limburg met één ambitie: een kinderwagen
ontwerpen die zowel duurzaam als gebruiksvriendelijk is. Het was het begin van
een reeks innovatieve ideeën die samenkwamen in een uniek product
\parencite{GreentomOnzeMisse}. Het product legt niet alleen de nadruk op
duurzame materialen, maar ook op de principes repareerbaarheid, veelzijdigheid
en modulariteit \parencite{GreentomDuurzaamheid}. Zo kan de kinderwagen
eenvoudig door de consument in slechts vijf minuten gebouwd worden, een
werkelijk reuzeverschil met de gemiddelde twee uur van concurrenten
\parencite{SmitsmansPresentatieNov2025}.

\section{B Corp-certificaat}
Greentoms toewijding aan duurzaamheid werd verder onderstreept toen het bedrijf
het B Corp-certificaat mocht ontvangen \parencite{GreentomBCorp}. Dit is een
certificaat dat wordt uitgereikt door B-lab, een non-profitorganisatie dat
bedrijven beoordeelt op hun maatschappelijke en ecologische impact
\parencite{BLabAboutUs}. Deze certificering vereist een grondige toetsing:
bedrijven moeten minstens 80 punten behalen in de B Impact Assessment, waarbij
wordt gekeken naar vijf kerngebieden: bestuur, medewerkers, gemeenschappen,
milieu en klanten \parencite{BLabWhat}.

%\section{Leasen en tweedehands}
