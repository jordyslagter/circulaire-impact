\chapter{Taken}

\section{Taak 6}
\newtheorem{questionanswer6}{Vraag}
\begin{questionanswer6}[Waarom is ‘sustainability leadership’ het leiderschap van de toekomst?]

  sustainability leadership focust in plaats van alleen op kortetermijnresultaten vooral op om in de lange termijn gangbaar te werkten.
  Hierdoor is het mogelijk om niet alleen nu maar ook in de toekomst te blijven bestaan.

\end{questionanswer6}
\begin{questionanswer6}[Met welke uitdagingen hebben leiders te maken?]

  Leiders moeten meerdere partijen tegelijk tevreden houden, maar partijen kunnen soms tegenstrijdige doelen hebben. 
  Zo zal bijvoorbeeld het doel van aandeelhouders om zo veel mogelijk winst te maken in strijd zijn tegen de wensen van medewerkers en het millieu.

\end{questionanswer6}
\begin{questionanswer6}[Je ziet nu dat leiders zich niet alleen tegenover hun aandeelhouders en de
buitenwereld moeten verantwoorden, maar ook tegenover hun eigen mensen. Waarom is dat?]

  Doordat medewerkers meer betrokken zijn in het bedrijf waarvor ze werken zullen ze meer geinteresseerd zijn om hun werk goed uit te voeren. 
  Hierdoor zal de productiviteit dus omhoog gaan.

\end{questionanswer6}
\begin{questionanswer6}[Volgens het onderzoek van Hofkes is er volgende de bestuurlijke top dringend
behoefte aan bestuurders die hun eigen belang ondergeschikt maken aan het
maatschappelijke belang. Nu zijn bestuurders naar eigen zeggen nog teveel
gericht op direct resultaat, korte-termijn belangen en winstgedrevenheid. Leg dit
uit? En hoe verklaar jij dat deze behoefte er is?]

  Het is traditioneel noodzakelijk voor aandeelhouders dat er zo snel mogelijk zo veel mogelijk geld opgeleverd wordt. 
  Omdat deze aandeelhouders invloed hebben op bedrijven houden bestuurders deze graag tevreden.
  Het zal echter noodzakelijk worden om meer na te denken over welke invloed keuzes op de lange termijn hebben en hoe deze de maatschappij en het millieu aantasten.

\end{questionanswer6}
\begin{questionanswer6}[Hoe verklaart Simon Sinek dat leiders inspireren tot actie?]

  Simon Sinek zegt dat je om als leider te inspireren je van 'binnenuit' met denken en uitleggen. 
  Dit houdt in dat je in plaats van eerst je product te laten zien en vervolgens dat product te verklaren je eerst de gedachten en filosofie achter het product uit moet leggen.

\end{questionanswer6}
\begin{questionanswer6}[Welke tip zou jij je opdrachtgever willen geven over leiderschap en hun kansen
en ambitiestatement na het bestuderen van de verplichte bronnen?
• Stel je persoonlijke mission statement op. Wat is jouw persoonlijke Why? Jouw
ambitie? Wat wil jij bereiken?]

  Wij denken dat het voor GreenTom goed zou zijn om beter hun missie en idialen te communiceren naar de klant.

\end{questionanswer6}
\begin{questionanswer6}[Stel je persoonlijke mission statement op. Wat is jouw persoonlijke Why? Jouw
ambitie? Wat wil jij bereiken?]

  Wij willen een toekomst zien waarin het mogelijk is om te leven zonder een nadelige invloed op de natuur. 
  Daarom willen we een volledige circulaire economie bereiken.

\end{questionanswer6}

\section{Taak 7}

\newtheorem{questionanswer7}{Vraag}
\begin{questionanswer7}[Wat is de essentie van de golden formula for sustainable business volgens
Melanie van der Velde?]

  Er moeten samenwerkingen ontstaan tussen tegenstellingen. Alleen zo kan je alle kanten tevreden houden.

\end{questionanswer7}
\begin{questionanswer7}[Wat was de duurzame ambitie bij Unilever? Welke doelen hadden zij?]

  De ambities van Unilever zijn onder andere om voor een werkzame loon te zorgen voor mensen in de productieketen, 
  een einde te brengen aan plastic vervuiling, een regeneratieve werking te hebben op de natuur en om tegen 2039 geen CO2 uit te stoten.

\end{questionanswer7}
\begin{questionanswer7}[Waar liep Paul Polman tegenaan in deze transitie?]

  Hij liep er tegenaan dat het moeilijk is om aandeelhouders te overtuigen van de transitie en dat het soms moeilijk is om te controleren, 
  vooral in het buitenland, of doelen daadwerkelijk worden behaald.

\end{questionanswer7}
\begin{questionanswer7}[Hoe verklaart Melanie van der Velde het gevecht bij Unilever?]

  Het probleem wat Melanie ziet bij Unilever is dat veel stappen die gezet moeten worden voor duurzaamheid niet genoeg geld opleveren voor Unilever om financieel haalbaar te zijn.

\end{questionanswer7}
\begin{questionanswer7}[Hoe hadden de acht veranderstappen van Kotter kunnen bijdragen bij Unilever?]

  Door gebruik te maken van de acht stappen van Kotter zou er binnen Unilever een duidelijker en beter gecommuniceerd toekomstbeeld zijn.
  Hierdoor zouden mederwerkers en aandeelhouders beter begrijpen welke veranderingen waarom gebeuren en is het dus gemakkelijker om steun hiervoor te krijgen.

\end{questionanswer7}
\begin{questionanswer7}[Wat is het succes van Interface, wat hebben zij anders gedaan dan Unilever?]

  Interface had een duidelijk doel en zijn duidelijk in hun communicatie van deze doelen en hun successen zowel richting medewerkers als klanten via de producten zelf.
  Door deze doelen en successen beter te communiceren is het duidelijk voor alle partijen waar het bedrijf naartoe gaat in de toekomst.

\end{questionanswer7}
\begin{questionanswer7}[Wat is de rol van een individu bij transitie volgens Jan Rotmans?]

  Individuele veranderingen zijn volgens Rotmans de enige weg naar systematische veranderingen.

\end{questionanswer7}
