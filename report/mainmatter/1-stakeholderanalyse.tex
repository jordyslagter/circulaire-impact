\chapter{Stakeholderanalyse}
Greentom wordt gekenmerkt door een sterke en helder geformuleerde missie.
De doelstellingen van het bedrijf zijn bewust eenvoudig gehouden, maar vormen
de kern van een bredere visie op duurzaamheid, functionaliteit en
maatschappelijke verantwoordelijkheid \parencite{GreentomOnzeMisse}.
Greentom staat voor ‘Green Tomorrow’ \parencite{GreentomOnzeMisse}. Het draagt
met deze naam de filosofie voort dat het de meest duurzame keuze is om geen
natuurlijke hulpbronnen meer te benutten \parencite{GreentomOnzeMisse}.

Greentom vervult deze doelstellingen door onder andere producten te ontwikkelen
die voor 87\% uit recyclebaar materiaal bestaan
\parencite{SmitsmansPresentatieNov2025}. Het verzorgt daarnaast ook
partnerschappen met lease-bedrijven waardoor consumenten haar producten niet
alleen kunnen kopen maar ook kunnen huren
\parencite{SmitsmansPresentatieNov2025}.

\section{Eerste wieltjes}
In 2013 startte Greentom in Zuid-Limburg met één ambitie: een kinderwagen
ontwerpen die zowel duurzaam als gebruiksvriendelijk is. Het was het begin van
een reeks innovatieve ideeën die samenkwamen in een uniek product
\parencite{GreentomOnzeMisse}. Het product legt niet alleen de nadruk op
duurzame materialen, maar ook op de principes repareerbaarheid, veelzijdigheid
en modulariteit \parencite{GreentomDuurzaamheid}. Zo kan de kinderwagen
eenvoudig door de consument in slechts vijf minuten gebouwd worden, een
werkelijk reuzeverschil met de gemiddelde twee uur van concurrenten
\parencite{SmitsmansPresentatieNov2025}.

\section{B Corp-certificaat}
Greentoms toewijding aan duurzaamheid werd verder onderstreept toen het bedrijf
het B Corp-certificaat mocht ontvangen \parencite{GreentomBCorp}. Dit is een
certificaat dat wordt uitgereikt door B-lab, een non-profitorganisatie dat
bedrijven beoordeelt op hun maatschappelijke en ecologische impact
\parencite{BLabAboutUs}. Deze certificering vereist een grondige toetsing:
bedrijven moeten minstens 80 punten behalen in de B Impact Assessment, waarbij
wordt gekeken naar vijf kerngebieden: bestuur, medewerkers, gemeenschappen,
milieu en klanten \parencite{BLabWhat}.

\section{Plaats in de keten}
Om de haalbaarheid van doelstellingen te bekrachigen is het belangrijk om de
plaats van Greentom in de keten te onderzoeken. Hier voor is gebruik gemaakt
van de Value Hill \parencite{Valuehill2016}.

De belangrijkste partijen in de keten zijn de leveranciers: rPET en rPP, en de
producenten van het frame en de stof. Deze leveranciers leveren gerecyclede
kunststoffen \parencite{SmitsmansMailNov2025, RPPLanding}. % mail guillaume
De rol van Greentom in de keten is om het product op een circulaire manier te
ontwerpen, te marketen en samenwerkingen aan te leggen met leveranciers,
producenten en andere derde partijen om hun producten op een circulaire manier
te fabriceren, marketen en distribueren
\parencite{SmitsmansPresentatieNov2025, SmitsmansMailNov2025}. % mail guillaume

Klanten spelen in de keten een belangrijke rol. Aangezien Greentom
onderdelen voor reparaties beschikbaar stelt kan een klant een onderdeel kopen
in het geval van defect, in plaats een nieuw product te hoeven kopen
\parencite{SmitsmansPresentatieNov2025}. Er zijn ook leveranciers van leenwagens
die Greentoms oplossingen als optie aanbieden. Door gebruik te maken van deze
services hoeft een klant niet eens een product te kopen, dit draagt bij aan het
verminderen van afval \parencite{SmitsmansPresentatieNov2025}.

Het product wordt verkocht via drie verkoopkanalen: distributeurs, waarnaar
minimaal een zeecontainer met 400 wordt verkocht, fysieke retailers en online
webshops (greentom.com, bol.com en amazon.nl). Bij de laatste 2 kunnen klanten
het product direct per stuk kopen \parencite{SmitsmansMailNov2025}. Het product
wordt hier verkocht aan jonge ouders in voornamelijk oost-Europa en China
\parencite{SmitsmansPresentatieNov2025}.

Aan het einde van de levensfase van het gekochte product kan 86\% er van
gerecycled worden \parencite{SmitsmansPresentatieNov2025}. In het geval van
lenen wordt het teruggebracht naar de dienstverlener
\parencite{SmitsmansPresentatieNov2025}.

Greentom gebruikt al gerecycled materiaal, waardoor er weinig waarde verloren
gaat in de productie, echter zou het nog verbeterd kunnen worden door hun
eigen producten weer te hergebruiken, refurbishen of zelf te recyclen
\parencite{SmitsmansMailNov2025}.

%\section{Leasen en tweedehands}
