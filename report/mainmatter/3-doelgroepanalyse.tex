\chapter{Doelgroepanalyse}
Om erachter te komen wat studenten zoeken in een rugzak is er een enquête uitgevoerd.

\section{Opzet}

\section{Resultaten}

\begin{figure}[h]
  \centering

  \begin{tikzpicture}
    \csvreader[]{data/enquete_duurzaamheid.csv}{Name = \Name, Quantity = \Quantity}{\Quantity/\Name,}
    %\pie{data/duurzaamheid_enquete.csv}
  \end{tikzpicture}

  \caption{Antwoorden op de vraag belang van duurzaamheid}
  \label{fig:duurzaamheid}
\end{figure}


\subsection{Demografie}
De enquête is beantwoord door 19 respondenten. Deze zijn voornamelijk Zuyd
Hogeschool-studenten met een leeftijd tussen de 18 en 23. De studenten
gebruiken bijna allemaal momenteel een rugzak. Een groot deel van deze
geeft aan tevreden te zijn met diens huidige rugzak, met 73\% van de scores een 8
of hoger.

\subsection{Rugzakkenmerken}
Uit de data blijkt dat respondenten willen dat een rugzak sowieso een laptopvak,
waterfleshouder, waterdichtheid, comfort, organisatie, een elektronica vakje,
anti-diefstal ritsen en design belangrijk vinden. Bovendien blinken een laptopvak,
waterdichtheid, comfort en organisatie uit als uitermate belangrijk. Verder
wordt er ook gesproken over het belang van kwaliteit en een lange levensduur.

\subsection{Duurzaamheid}
Geen enkel van de respondenten geeft aan duurzaamheid ‘heel belangrijk’ te
vinden bij het kopen van een nieuw product.
Een derde geeft aan helemaal niet op duurzaamheid te letten.

\section{Discussie}
De gegevens geven aan dat de doelgroep voor \( \frac{2}{3}^{\text{de}} \) weinig tot geen
aandacht besteedt aan de duurzaamheid van producten die zij kopen, de
enquête ging echter over Zuyd Hogeschool-studenten, een reeks hogescholen in
zuid Limburg. Zuid Limburg is een regio waarbij vooral partijen die weinig
willen werken aan de circulariteit van Nederland, of deze zelfs willen
terugdraaien worden verkozen \parencite{NOSTK2025, PVV2025}.

\section{Conclusie}
In conclusie blijkt dat er onder Zuid-Limburgse studenten niet veel gelet
wordt op duurzaamheid bij het kopen van een rugzak. Wel wordt er gelet op
kwaliteit en bruikbaarheid. Het zou daarom dus meer lonen om bij de marketing
naast duurzaamheid ook een focus te leggen op deze factoren.

