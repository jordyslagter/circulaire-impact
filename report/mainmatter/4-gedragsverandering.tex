\chapter{Gedragsverandering}
Zoals beschreven in \autoref{chap:doelgroepanalyse} is het van groot belang dat
er een gedragsverandering zal plaatsvinden in onze doelgroep. Hiervoor is het
model van de luiheidsvlieg gebruikt.

\section{Waarom niet?}
Mensen zullen niet naar een duurzame rugzak overstappen omdat het duurzaam is,
zoals bewezen in \autoref{chap:doelgroepanalyse}. Als er verder geen
aantrekkelijke eigenschappen van het product zijn zullen mensen het dus ook
niet gaan gebruiken.

\section{Hoe maken we de rugzak aantrekkelijker?}
Het idee is om de rugzak modulair te maken; vergelijkbaar met Framework.
De rugzak zal in het binnenste en het buitenste ‘loops’ hebben waaraan
nieuwe gedeeltes of vakjes bevestigd kunnen worden. Dit is vergelijkbaar
met een rugzak van Defensie.
