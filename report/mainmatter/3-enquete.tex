\chapter{Enquête}
Om erachter te komen wat studenten zoeken in een rugzak is er een enquête uitgevoerd. 

\section{Opzet}

\section{Resultaten}
\subsection{Demografie}
De enquête is beantwoord door 19 mensen. Deze mensen zijn voornamelijkZuyd Hogeschool studenten met een leeftijd tussen de 18 en 23.
Deze studenten bijna allemaal aan momenteel een rugzak te gebruiken.
Een groot deel van deze mensen geeft aan tevreden te zijn met hun huidige rugzak, met 73\% van de scores een 8 of hoger.

\subsection{Rugzakkenmerken}
Uit de data blijkt dat studenten willen dat een rugzak sowieso een laptopvak, waterfleshouder, waterdichtheid, comfort, organisatie, een elektronica vakje, anti-diefstal ritsen en design belangrijk vinden.
Hie springen een laptopvak, waterdichtheid, comfort en organisatie uit als uitermate belangrijk.
Verder wordt er ook gesproken over het belang van kwaliteit en een lange levernsduur.

\subsection{Duurzaamheid}
Er wordt door studenten niet veel gelet op duurzaamheid bij het kopen van een nieuw product.
Een derde van de mensen geeft zelfs aan helemaal niet op duurzaamheid te letten.
